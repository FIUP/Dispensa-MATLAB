\chapter{Utility generali}
All'interno di MATLAB esistono vari comandi che aiutano l'utente a districarsi tra le varie sue funzionalità.
Tra i più importanti ricordiamo sicuramente:

\begin{itemize}
	
	\item 	\textbf{help} \\ 
		  	Questo comando permette di visualizzare la schermata di aiuto per una data funzionalità, e può essere 
		  	utilizzato in due modi differenti:
		  	\begin{enumerate}
		  	
				\item	\texttt{help}: elenca tutti i topic principali di aiuto nella Finestra Comandi. Ogni topic
						principale corrisponde al nome di una cartella nel percorso di ricerca di MATLAB.
						
				\item	\texttt{help \textcolor{blue}{nome}}: visualizza il testo di aiuto per la funzionalità 
						specificata da \texttt{\textcolor{blue}{nome}}, come una funzione, un metodo, una classe, 
						una toolbox o una variabile.
		  	
		  	\end{enumerate}
		  	
	\item	\textbf{lookfor} \\
			Questo comando viene utilizzato quando si vuole trovare il nome di una funzione che svolga un dato
			compito. Viene utilizzata nei seguenti modi:
			\begin{enumerate}
			
				\item 	\texttt{lookfor \textcolor{blue}{topic}}: ricerca la stringa \textcolor{blue}{topic}
						all'interno della prima riga di commento (H1) dei programmi MATLAB trovati all'interno del
						percorso di ricerca. Per ogni file per il quale si è trovata una corrispondenza, MATLAB
						mostra l'intera riga H1.
						
				\item 	\texttt{lookfor \textcolor{blue}{topic} \textcolor{red}{-all}}: ricerca una corrispondenza 
						per la stringa \textcolor{blue}{topic} nell'\textcolor{red}{intero} primo blocco di 
						commento dei programmi nel percorso di ricerca.
			
			\end{enumerate}
	
\end{itemize}