\chapter{Utility numeriche}
I comandi più utili dal punto di vista puramente numerico sono i seguenti.

\section{Costanti macchina}
\begin{itemize}

	\item	\texttt{realmin}: mostra il più piccolo numero macchina rappresentabile. \\
			Valore: \texttt{2.2250738585072 e-308}
	
	\item 	\texttt{realmax}: mostra il più grande numero macchina rappresentabile. \\
			Valore: \texttt{1.79769313486232 e+308}
			
	\item	\texttt{eps}: mostra la precisione di macchina, ovvero la più piccola quantità di modulo diverso da 0
			che, sommato ad 1, restituisce un risultato diverso da 1. Esso è anche il più grande errore relativo
			compiuto nell'approssimazione di un numero reale all'interno dell'intervallo \texttt{[realmin, 
			realmax]}. \\
			Valore: \texttt{2.2204 e-16}

\end{itemize}

\section{Costanti numeriche}
\begin{itemize}

	\item 	\texttt{pi}: rappresenta l'approssimazione di $\pi$.
	
	\item 	\texttt{Inf}: rappresenta l'approssimazione di $\infty$.
			
	\item	\texttt{NaN}: Not a Number, rappresenta un risultato non computabile, come quello derivante da una
			espressione come $0/0$ o qualsiasi altra \textcolor{blue}{forma indeterminata}.
		
\end{itemize}

\section{Operazioni numeriche}

\paragraph{Operazioni aritmetiche}
\begin{itemize}

	\item	\texttt{+} : somma.
	
	\item	\texttt{-} : differenza.
	
	\item	\texttt{*} : prodotto.
	
	\item	\texttt{/} : quoziente.
	
	\item	\texttt{\^} : potenza.
	
\end{itemize}	
	
	
\paragraph{Funzioni elementari}
Sia \texttt{x} una qualsiasi espressione, costante, o un qualsiasi vettore. Allora in MATLAB esistono le seguenti.
\begin{itemize}	
	
	\item	\texttt{abs(x)} : valore assoluto.
	
	\item	\texttt{sin(x)} : seno.
	
	\item	\texttt{cos(x)} : coseno.
	
	\item	\texttt{tan(x)} : tangente.
	
	\item	\texttt{cot(x)} : cotangente.
	
	\item	\texttt{asin(x)} : arco seno.
	
	\item	\texttt{acos(x)} : arco coseno.
	
	\item	\texttt{atan(x)} : arco tangente.
	
	\item	\texttt{sinh(x)} : seno iperbolico.
	
	\item	\texttt{cosh(x)} : coseno iperbolico.
	
	\item	\texttt{tanh(x)} : tangente iperbolica.
	
	\item	\texttt{asinh(x)} : arco seno iperbolico.
	
	\item	\texttt{acosh(x)} : arco coseno iperbolico.
	
	\item	\texttt{atanh(x)} : arco tangente iperbolica.
	
	\item	\texttt{sqrt(x)} : radice quadrata.
	
	\item	\texttt{exp(x)} : esponenziale.
	
	\item	\texttt{log2(x)} : logaritmo in base 2.

	\item	\texttt{log10(x)} : logaritmo in base 10.
	
	\item	\texttt{log(x)} : logaritmo naturale (base $e$).
	
	\item	\texttt{fix(x)} : arrotondamento verso lo 0.
	
	\item	\texttt{round(x)} : arrotondamento verso l'intero più vicino (per eccesso o difetto).
	
	\item	\texttt{floor(x)} : arrotondamento verso $-\infty$ (per difetto).
	
	\item	\texttt{ceil(x)} : arrotondamento verso $+\infty$ (per eccesso).
	
	\item	\texttt{sign(x)} : segno \\ 
	Valore: \texttt{-1} se l'elemento è $< 0$,  \texttt{0} se l'elemento è $0$,  \texttt{+1} se l'elemento è $> 
	0$).
	
	\item	\texttt{rem(x)} : resto di una divisone.

\end{itemize}