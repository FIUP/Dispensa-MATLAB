\chapter{Formati di visualizzazione dei numeri}
\section{Come MATLAB memorizza i numeri}
MATLAB dispone di vari formati numerici che visualizzano, quando necessario, i numeri macchina in modi diversi:
\begin{itemize}

	\item	\texttt{format \textcolor{blue}{short}}: notazione decimale con 4 cifre dopo la virgola.
	
	\item	\texttt{format \textcolor{blue}{short} \textcolor{red}{e}}: notazione \textcolor{red}{esponenziale} 
			con 4 cifre dopo la virgola.
			
	\item	\texttt{format \textcolor{blue}{long}}: decimale con 15 cifre dopo la virgola in doppia precisione, e 
			7 cifre dopo la virgola in singola precisione.
			
	\item	\texttt{format \textcolor{blue}{long} \textcolor{red}{e}}: notazione \textcolor{red}{esponenziale} 
			con 15 cifre dopo la virgola in doppia precisione, e 7 cifre dopo la virgola in singola precisione.
			
	\item	\texttt{format \textcolor{blue}{long} \textcolor{red}{g}}: la più compatta tra \texttt{format long} e 
			\texttt{format long e}.

\end{itemize}

Oltre a questi esistono due comandi per trasformare i numeri da doppia a singola precisione e viceversa:
\begin{itemize}

	\item	\texttt{single(x)}: converte un numero in precisione singola. Può causare inconsistenze, meglio usare 
	\textit{double(single(pi))}.
	
	\item 	\texttt{double(x)}: converte un numero in precisione doppia.
\end{itemize}

Esempio: \texttt{format(long)}: cambia la formattazione, come il numero di cifre dopo la virgola.
