\chapter{Istruzioni condizionali, cicli for e while}
Come all'interno di molti linguaggi di programmazione, anche nel linguaggio MATLAB è possibile utilizzare le istruzioni 
condizionali e i cicli for e while.

\section{Istruzioni condizionali}
L'istruzione condizionale principalmente utilizzata è l'istruzione \texttt{if}: \break

\begin{center}
\texttt{if } \textit{espressione} \\ 
\hspace{1.5cm}\textit{istruzioni}  \\
\hspace{1cm}\texttt{elseif } \textit{espressione} \\ 
\hspace{1.4cm}\textit{istruzioni}  \\
\hspace{-1.8cm}\texttt{else} \\ 
\hspace{1.4cm}\textit{istruzioni}  \\
\hspace{-2cm}\texttt{end}
\end{center}

Oltre all'istruzione  \texttt{if} esiste anche quella di \texttt{switch}:

\begin{center}
\texttt{switch } \textit{espressione\_di\_switch} \\ 
\hspace{1cm}\texttt{case } \textit{espressione\_di\_case} \\ 
\hspace{1.4cm}\textit{istruzioni}  \\
\hspace{1cm}\texttt{case } \textit{espressione\_di\_case} \\ 
\hspace{1.4cm}\textit{istruzioni}  \\
\hspace{2cm} \texttt{...} \\
\hspace{-2cm}\texttt{otherwise } \\ 
\hspace{1.4cm}\textit{istruzioni}  \\
\hspace{-5cm}\texttt{end}
\end{center}

Gli operatori di confronto sono i seguenti:
\begin{itemize}

	\item	\texttt{== }: uguale.
	\item	\texttt{\textasciitilde= }: non uguale.
	\item	\texttt{< }: minore.
	\item	\texttt{> }: maggiore.
	\item	\texttt{<= }: minore o uguale.
	\item 	\texttt{>= }: maggiore o uguale.

\end{itemize}

Inoltre più espressioni logiche possono essere combinate tra loro mediante i seguenti:
\begin{itemize}

	\item	\texttt{\&\& }: and.
	\item	\texttt{|| }: or.
	\item	\texttt{\textasciitilde }: not
	\item	\texttt{\& }: and componente per componente.
	\item	\texttt{| }: or componente per componente.

\end{itemize}


\section{Cicli \texttt{for} e \texttt{while}}
\subsection{Il ciclo \texttt{for}}
Il ciclo \texttt{for} itera una porzione di codice, al variare di certi indici. La sua scrittura è la seguente:
\begin{center}
\texttt{for } \textit{variabile} \texttt{=} \textit{vettore} \\ 
\hspace{1.5cm}\textit{istruzioni}  \\
\hspace{-3.5cm}\texttt{end}
\end{center}

Esempio: \\
\texttt{>> s=0;} \\
\texttt{for } \texttt{j = 1:10} \\ 
\hspace{1cm}\texttt{s=s+j;}  \\
\texttt{end}

\subsection{Il ciclo \texttt{while}}
Il ciclo \texttt{while}, a differenza di quello \texttt{for}, viene utilizzato quando \textbf{non è noto a priori} il 
numero di volte che il ciclo dovrà iterare. \\
La sua scrittura è la seguente:
\begin{center}
\texttt{while } \textit{espressione\_logica} \\ 
\hspace{1.5cm}\textit{istruzioni}  \\
\hspace{-4.1cm}\texttt{end}
\end{center}

Esempio: \\
\texttt{>> s=1;} \\
\texttt{while } \texttt{s>0} \\ 
\hspace{1cm}\texttt{s=rand(1)-0.5;}  \\
\texttt{end}