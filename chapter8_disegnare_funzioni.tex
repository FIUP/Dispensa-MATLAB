\chapter{Disegnare le funzioni}
\section{Il comando \texttt{plot}}
All'interno di MATLAB è possibile disegnare il grafico delle funzioni, attraverso il comando \texttt{plot}.
La segnatura di questo comando è la seguente:
$$ \texttt{plot(X, Y)} $$
Questo crea un grafico 2D dei dati in \texttt{Y} contro i corrispondenti dati in \texttt{X}.
\begin{itemize}
	\item 	Se \texttt{X} e \texttt{Y} sono entrambi vettori, devono avere lunghezza uguale. Disegnerà Y contro X.
	\item	Se \texttt{X} e \texttt{Y} sono entrambi matrici, allora devono avere dimensione uguale. La funzione
			disegnerà le colonne di \texttt{Y} contro le colonne di \texttt{X}.
	\item	Se uno tra \texttt{X} e \texttt{Y} è un vettore e l'altro una matrice, allora la matrice deve avere
			dimensioni tali per cui una delle sue dimensioni è uguale alla lunghezza del vettore. \\
			Se il numero di righe della matrice è uguale alla lunghezza del vettore, allora \texttt{plot} 
			disegnerà ogni colonna della matrice contro il vettore. Se il numero di colonne della matrice è uguale
			alla lunghezza del vettore, disegnerà invece ogni riga della matrice contro il vettore. \\
			Se la matrice è quadrata, allora disegnerà ogni colonna contro il vettore.
	\item	Se uno tra \texttt{X} e \texttt{Y} è uno scalare e l'altro è una matrice o un vettore, allora disegna
			dei punti. Per vedere tali punti bisogna però specificare un simbolo di marcatura, per esempio,
			\texttt{plot(X, Y, 'o'}.
\end{itemize}

Oltre a questo, con il comando $$\texttt{plot(X, Y, LineSpec)}$$ è possibile impostare delle lo stile, il tratto e il 
colore della linea. \break \\
Con $$\texttt{plot(X1,Y1,...,Xn,Yn)}$$ è invece possibile disegnare diverse coppie \texttt{X}, \texttt{Y} utilizzando 
gli stessi assi per le varie linee. \break \\

\textcolor{Red}{\textbf{N.B.}} Per vedere tutte le opzioni della funzione \texttt{plot} è utile il comando \texttt{help 
plot}.