\chapter{Le funzioni definite dall'utente}
\section{Definire una funzione}
In MATLAB un utente può definire una funzione scrivendo un M-file, ovvero un file con estensione \textit{.m}.
Per creare una funzione la sintassi da utilizzare all'interno del file \textit{.m} è la seguente:

$$  \texttt{function [y1, ..., yN] = myfunc(x1, ..., xN)}$$

Dove:
\begin{enumerate}
	\item	\texttt{function} è una keyword obbligatoria.
	\item	\texttt{y1, ..., yN} sono i parametri di output della funzione, che possono avere nomi arbitrari.
	\item 	\texttt{myfunc} è il nome della funzione che si vuole creare.
	\item	\texttt{x1, ..., xN} sono i parametri di input della funzione, che possono avere nomi arbitrari. 
\end{enumerate}

Esempio, funzione a due parametri anche ritornati:\\

\texttt{>> function [s , t ] = fun2(x , y);} \\
\texttt{>> s=(x+y);} \\
\texttt{>> t=(x-y);} \\ 

Una volta scritta questa intestazione il corpo della funzione viene scritto sotto di essa. \\
Per richiamare la funzione definita, è sufficiente utilizzare la stessa firma senza la keyword \texttt{function}. 
\break \break
\textbf{N.B.} Si è soliti salvare la funzione in un file \textit{.m} con lo stesso nome della funzione. 

Senza ricorrere a nuovi files si può definire con il comando di definizione di funzione matematica @
Esempio: \\
\texttt{>> f=@(x) sin (x)+pi;} \\
\texttt{f(0) } \\
\texttt{feval(f ,0) } \% in alternativa \\
\texttt{ans = } \\
\texttt{\hspace{0.5cm} 3.1416} \break