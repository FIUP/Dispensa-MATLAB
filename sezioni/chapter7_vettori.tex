\chapter{I vettori}
All'interno di MATLAB vengono spesso utilizzati i \textbf{vettori}, a tal punto che quasi tutte le funzioni e le 
operazioni disponibili sono definite anche su essi. Vediamo quindi come poter creare e trattare i vettori.

\section{Creare un vettore riga}
Per creare un vettore \textbf{riga} è sufficiente utilizzare la seguente segnatura:
$$ \texttt{v = [x1 ... xN]} $$
dove \texttt{x1, ..., xN} sono le componenti che si vogliono inserire nel vettore.

Esempio: \\
\texttt{>> v = [1 2 3]} \\
\texttt{v = } \\
\texttt{\hspace{0.5cm} 1 \hspace{0.5cm} 2 \hspace{0.5cm} 3} \\
\texttt{>>}

\section{Aggiungere elementi ad un vettore}
Dato \texttt{v = [x1 ... xN]} un vettore qualsiasi, per aggiungere elementi a v si utilizza la seguente scrittura:
$$ \texttt{v = [v xN+1 ... xM]} $$
dove \texttt{xN+1, ..., xM} sono le componenti aggiuntive che si vogliono inserire.

Esempio: \\
\texttt{>> v = [1 2 3];} \\
\texttt{>> v = [v 9 10]} \\
\texttt{v = } \\
\texttt{\hspace{0.5cm} 1 \hspace{0.5cm} 2 \hspace{0.5cm} 3 \hspace{0.5cm} 9 \hspace{0.5cm} 10} \\
\texttt{>>}

\newpage

\section{Selezionare gli elementi di un vettore}
Per selezionare il \textbf{j-esimo} elemento di un vettore \texttt{v = [x1 ... xN]} qualsiasi si utilizza la seguente 
segnatura:
$$ \texttt{v(j)} $$

Esempio: \\
\texttt{>> v = [5 4 9]} \\
\texttt{v = } \\
\texttt{\hspace{0.5cm} 5 \hspace{0.5cm} 4 \hspace{0.5cm} 9} \\
\texttt{>> v(2)} \\
\texttt{ans = } \\
\texttt{\hspace{0.5cm} 4} \break

Per selezionare invece l'\textbf{ultima} componente di un vettore si utilizza
$$ \texttt{v(end)} $$

Esempio: \\
\texttt{>> v = [5 4 9]} \\
\texttt{v = } \\
\texttt{\hspace{0.5cm} 5 \hspace{0.5cm} 4 \hspace{0.5cm} 9} \\
\texttt{>> v(end)} \\
\texttt{ans = } \\
\texttt{\hspace{0.5cm} 9} \break

Per conoscere la lunghezza di un vettore, il comando da utilizzare è:
$$\texttt{length(v)} $$

Esempio: \\
\texttt{>> v = [5 4 9]} \\
\texttt{v = } \\
\texttt{\hspace{0.5cm} 5 \hspace{0.5cm} 4 \hspace{0.5cm} 9} \\
\texttt{>> length(v)} \\
\texttt{ans = } \\
\texttt{\hspace{0.5cm} 3} \break


Per definire una matrice, cioè un vettore a più dimensioni possiamo usare la seguente sintassi:

\texttt{>>matrice = [1,2,3;4,5,6;7,8,9]} \\

Per conoscere la lunghezza di un vettore a pù dimensioni, il comando da utilizzare è:
$$\texttt{size(v)} $$

Esempio usando il vettore v definito in precedenza: 
\texttt{>> size(v)} \\
\texttt{ans = } \\
\texttt{\hspace{0.5cm} 1 \hspace{0.5cm} 3} \break
Dove il primo valore indica il numero delle righe e il secondo quello delle colonne.

\section{Creare un vettore colonna}
Per creare un vettore \textbf{colonna} si usa la seguente firma:
$$ \texttt{v = [x1 ; ... ; xN]} $$
dove \texttt{x1, ..., xN} sono le componenti da inserire nel vettore. \\
Alternativamente, si può creare un vettore colonna partendo da un vettore riga tramite la funzione di 
\textbf{trasposizione}. \\
Sia \texttt{x = [x1 ... xN]} un vettore riga, allora 
$$ y = v' $$
crea il vettore \texttt{y = [x1 ; ... ; xN]}, ovvero la trasposizione di x.

\section{Creare un vettore equi-spaziato}
Per creare un vettore \texttt{v} equi-spaziato in MATLAB è disponibile il seguente comando:
$$ \texttt{v = (a:h:b)} $$
Dove:
\begin{itemize}
	\item 	\texttt{a} è il punto iniziale.
	\item	\texttt{h} è la spaziatura.
	\item	\texttt{b} è il punto finale.
\end{itemize}

Se invece è noto il punto di inizio \texttt{a}, il punto di fine \texttt{b} e il numero di punti totali \texttt{N} si 
può utilizzare il comando seguente:
$$ \texttt{v = linspace(a,b,N)} $$

Consente di implementare un vettore con un numero di elementi equispaziati pari a fine - inizio (b-a)\\
Esempio:
\texttt{>> x = linspace(0,1,5)} \\
\texttt{ans = } \\
\texttt{\hspace{0.5cm} x = 0 \hspace{0.5cm} 0.25 \hspace{0.5cm} 0.5 \hspace{0.5cm} 0.75 \hspace{0.5cm} 1.0} \break

\section{Operazioni con i vettori}
All'interno di MATLAB sono definite le seguenti funzioni elementari con i vettori:
\begin{itemize}

	\item	\texttt{+ }: addizione
	\item	\texttt{- }: sottrazione
	\item	\texttt{.* }: prodotto puntuale, fa il prodotto elemento per elemento.
	\item	\texttt{./ }: quoziente puntuale , fa la divisione elemento per elemento.
	\item	\textbf{.\^}: potenza puntuale , eleva a potenza elemento per elemento.

\end{itemize}

Oltre ad esse sono definite anche tutte le funzioni elementari viste con gli scalari nella \textbf{Sezione 2.3}.
