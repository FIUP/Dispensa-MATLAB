\chapter{Disegnare le funzioni}

\section{Il comando \texttt{plot}}
All'interno di MATLAB è possibile disegnare il grafico delle funzioni, attraverso il comando \texttt{plot}.
La firma di questo comando è la seguente:
$$ \texttt{plot(X, Y)} $$
Questo crea un grafico 2D dei dati in \texttt{Y} contro i corrispondenti dati in \texttt{X}.
\begin{itemize}
	\item 	Se \texttt{X} e \texttt{Y} sono entrambi vettori, devono avere lunghezza uguale. Disegnerà Y contro X.
	\item	Se \texttt{X} e \texttt{Y} sono entrambi matrici, allora devono avere dimensione uguale. La funzione
			disegnerà le colonne di \texttt{Y} contro le colonne di \texttt{X}.
	\item	Se uno tra \texttt{X} e \texttt{Y} è un vettore e l'altro una matrice, allora la matrice deve avere
			dimensioni tali per cui una delle sue dimensioni è uguale alla lunghezza del vettore. \\
			Se il numero di righe della matrice è uguale alla lunghezza del vettore, allora \texttt{plot} 
			disegnerà ogni colonna della matrice contro il vettore. Se il numero di colonne della matrice è uguale
			alla lunghezza del vettore, disegnerà invece ogni riga della matrice contro il vettore. \\
			Se la matrice è quadrata, allora disegnerà ogni colonna contro il vettore.
	\item	Se uno tra \texttt{X} e \texttt{Y} è uno scalare e l'altro è una matrice o un vettore, allora disegna
			dei punti. Per vedere tali punti bisogna però specificare un simbolo di marcatura, per esempio,
			\texttt{plot(X, Y, 'o')}.
\end{itemize}

Oltre a questo, con il comando $$\texttt{plot(X, Y, LineSpec)}$$ è possibile impostare lo stile, il tratto e il 
colore della linea. \break \\
Con $$\texttt{plot(X1,Y1,...,Xn,Yn)}$$ è invece possibile disegnare diverse coppie \texttt{X}, \texttt{Y} utilizzando 
gli stessi assi per le varie linee. \break \\

\textcolor{Red}{\textbf{N.B.}} Per vedere tutte le opzioni della funzione \texttt{plot} è utile il comando 
\texttt{help plot}.

Per sovrapporre più grafici, si usano i comandi \texttt{holdon} (mostra più curve alla volta) e \texttt{holdoff}, 
intervallati dai plot da sovrapporre.

\section{Altre funzioni utili}

\begin{itemize}
    \item \texttt{xlabel('etichetta')}: etichetta l'asse delle x
    \item \texttt{ylabel('etichetta')}: etichetta l'asse delle y
    \item \texttt{title}: da un titolo al grafico
    \item \texttt{axis equal}: non scala il grafico
    \item \texttt{semilogy(x2, y2, 'g+')}: al posto di plot utilizza una scala logaritmica in base 10 per l'asse delle y
    \item \texttt{subplot(m,n,k)}: m righe per n colonne; disegna m*n grafici, k indica che le istruzioni che seguono 
    si riferiscono al kappesimo grafico. Si parte a contare dal primo in alto a sinistra, verso destra.
\end{itemize}

\section{Disegnare grafici 3D}
I grafici in MATLAB sono visti come superfici, per rappresentarli serve una griglia di valori a partire da x e y.
L'istruzione seguente trasforma i valori di x e y in array utilizzabili per generare plot 3D. \\
 
Esempio:\\
\texttt{[X,Y] = meshgrid(x,y)}\\
\texttt{surf(X,Y,Z)}

\paragraph{Personalizzare un grafico 3D}

\begin{itemize}
    \item \texttt{view}: consente di modificare l'orientamento del grafico.
    \item \texttt{colormap}: consente di definire il colore del grafico.
    \item \texttt{shading}: permette di cambiare le impostazioni di ombreggiatura.
    \item \texttt{mesh}: disegna un grafico a griglia.
    \item \texttt{contour}: curve di livello 2d visto dall'alto.
\end{itemize}