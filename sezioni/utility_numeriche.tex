\chapter{Utility numeriche}

\section{Tipi e Conversioni}

\paragraph{Tipi principali}

I tipi principali di MATLAB sono:

\begin{itemize}
    \item	\texttt{logical}: valore boolean.
    \item	\texttt{char}: ad es. 'a', carattere.
    \item	\texttt{uint8}: 1 byte, rappresenta interi senza segno nel range 0-255.
    \item	\texttt{uint16}: 2 byte, rappresenta interi senza segno nel range 0-65535.
    \item	\texttt{uint32}: 4 byte, 
    \item	\texttt{int8}: 1 byte, rappresenta interi con segno nel range -128-127. 
    \item	\texttt{int16}: 2 byte.
    \item	\texttt{int32}: 4 byte.
    \item	\texttt{single}: 4 byte, rappresenta valori float.
    \item	\texttt{double}: 8 byte, rappresenta valori double.
\end{itemize}

I nomi dei tipi sono anche delle utili funzioni per effettuare dei cast, ad esempio:\\
\texttt{>> a = uint(8)} \\
\texttt{>> c = double(a)} \\
\texttt{>> b = logical(a)} \\
\texttt{b = } \\
\texttt{\hspace{0.5cm} 1} \break

I comandi più utili dal punto di vista puramente numerico sono i seguenti.

\newpage

\section{Costanti macchina}
\begin{itemize}

	\item	\texttt{realmin}: mostra il più piccolo numero macchina rappresentabile. \\
			Valore: \texttt{2.2250738585072 e-308}
	
	\item 	\texttt{realmax}: mostra il più grande numero macchina rappresentabile. \\
			Valore: \texttt{1.79769313486232 e+308}
			
	\item	\texttt{eps}: mostra la precisione di macchina, ovvero la più piccola quantità di modulo diverso da 0
			che, sommato ad 1, restituisce un risultato diverso da 1. Esso è anche il più grande errore relativo
			compiuto nell'approssimazione di un numero reale all'interno dell'intervallo \texttt{[realmin, 
			realmax]}. \\
			Valore: \texttt{2.2204 e-16}

\end{itemize}

\section{Costanti numeriche}
\begin{itemize}

    \item 	\texttt{ans}: se non assegnato contiene il risultato dell'ultima operazione effettuata.
    
    \item 	\texttt{Inf}: rappresenta l'approssimazione di $\infty$.
    
    \item 	\texttt{i}: radice quadrata di -1.
        
    \item 	\texttt{j}: radice quadrata di -1.
    
	\item 	\texttt{pi}: rappresenta l'approssimazione di $\pi$.
			
	\item	\texttt{NaN}: Not a Number, rappresenta un risultato non computabile, come quello derivante da una
			espressione come $0/0$ o qualsiasi altra \textcolor{blue}{forma indeterminata}.
		
\end{itemize}

\section{Operazioni}

\paragraph{Operazioni aritmetiche}
\begin{itemize}

	\item	\texttt{+} : somma.
	
	\item	\texttt{-} : differenza.
	
	\item	\texttt{*} : prodotto.
	
	\item	\texttt{/} : quoziente.
	
	\item	\texttt{\^} : potenza.
	
\end{itemize}	
	
	
\paragraph{Funzioni elementari}
Sia \texttt{x} una qualsiasi espressione, costante, o un qualsiasi vettore. Allora in MATLAB esistono le seguenti.
\begin{itemize}	
	
	\item	\texttt{abs(x)} : valore assoluto.
	
	\item	\texttt{sin(x)} : seno.
	
	\item	\texttt{cos(x)} : coseno.
	
	\item	\texttt{tan(x)} : tangente.
	
	\item	\texttt{cot(x)} : cotangente.
	
	\item	\texttt{asin(x)}: arco seno.
	
	\item	\texttt{acos(x)}: arco coseno.
	
	\item	\texttt{atan(x)}: arco tangente.
	
	\item	\texttt{sinh(x)}: seno iperbolico.
	
	\item	\texttt{cosh(x)}: coseno iperbolico.
	
	\item	\texttt{tanh(x)}: tangente iperbolica.
	
	\item	\texttt{asinh(x)}: arco seno iperbolico.
	
	\item	\texttt{acosh(x)}: arco coseno iperbolico.
	
	\item	\texttt{atanh(x)}: arco tangente iperbolica.
	
	\item	\texttt{sqrt(x)}: radice quadrata.
	
	\item	\texttt{exp(x)}: esponenziale.
	
	\item	\texttt{log2(x)}: logaritmo in base 2.

	\item	\texttt{log10(x)}: logaritmo in base 10.
	
	\item	\texttt{log(x)}: logaritmo naturale (base $e$).
	
	\item	\texttt{fix(x)}: arrotondamento verso lo 0.
	
	\item	\texttt{round(x)}: arrotondamento verso l'intero più vicino (per eccesso o difetto).
	
	\item	\texttt{floor(x)}: arrotondamento verso $-\infty$ (per difetto).
	
	\item	\texttt{ceil(x)}: arrotondamento verso $+\infty$ (per eccesso).
	
	\item	\texttt{sign(x)}: segno \\ 
	Valore: \texttt{-1} se l'elemento è $< 0$,  \texttt{0} se l'elemento è $0$,  \texttt{+1} se l'elemento è $> 
	0$).
	
	\item	\texttt{rem(x)}: resto di una divisone.

    \item	\texttt{zeros(dim)}: crea una matrice di dim*dim zeri.
    \item	\texttt{rand(dim)}: crea una matrice di dim*dim valori casuali. 
    \item	\texttt{eye(5)}: crea una matrice 5X5 con in diagonale tutti 1.
    \item	\texttt{eye(2,3)}: crea una matrice 2x3 con in diagonale 1.
    \item	\texttt{eye(size(A))}: crea una matrice diagonale con in diagonale tutti 1, di dimensione come A.
    \item	\texttt{ones(3)}: crea una matrice 3x3 tutti 1.

    \item	\texttt{real(complex)}: permette di estrapolare la parte reale di un numero complesso.
    \item	\texttt{imag(complex)}: permette di estrapolare la parte immaginaria di un numero complesso.
    \item	\texttt{conj(complex)}: permette di calcolare il complesso coniugato.
    \item	\texttt{magic(n)}: costruisce un quadrato magico, n per n.
    \item	\texttt{prod(a)}: fa il prodotto degli elementi dell'array a.

\end{itemize}